\documentclass[12pt]{report}
\usepackage[top=1in, bottom=1in, left=0.5in, right=0.5in]{geometry}
\usepackage{amsmath, amsfonts, amssymb, libris, enumerate, fancyhdr, xcolor, lipsum, titling, minted, algorithm, algpseudocode, algorithmicx, float, hyperref, booktabs, graphicx, array}
\usepackage[utf8]{inputenc}

\usepackage[Glenn]{fncychap}
\usepackage[skip=20pt, indent=30pt]{parskip}
\usepackage{setspace}
\onehalfspacing

\usepackage[some]{background}

\definecolor{titlepagecolor}{cmyk}{.02,.04,0,.90}

\DeclareFixedFont{\bigsf}{T1}{phv}{b}{n}{1.5cm}
% Default fixed font does not support bold face
\DeclareFixedFont{\ttb}{T1}{txtt}{bx}{n}{12} % for bold
\DeclareFixedFont{\ttm}{T1}{txtt}{m}{n}{12}  % for normal

% Custom colors
\usepackage{color}
\definecolor{deepblue}{rgb}{0,0,0.5}
\definecolor{deepred}{rgb}{0.6,0,0}
\definecolor{deepgreen}{rgb}{0,0.5,0}

\usepackage{listings}

% Python style for highlighting
\newcommand\pythonstyle{\lstset{
language=Python,
basicstyle=\ttm,
morekeywords={self},              % Add keywords here
keywordstyle=\ttb\color{deepblue},
emph={MyClass,__init__},          % Custom highlighting
emphstyle=\ttb\color{deepred},    % Custom highlighting style
stringstyle=\color{deepgreen},
frame=tb,                         % Any extra options here
showstringspaces=false,
tabsize=2,
}}


% Python environment
\lstnewenvironment{python}[1][]
{
\pythonstyle
\lstset{#1}
}
{}

% Python for external files
\newcommand\pythonexternal[2][]{{
\pythonstyle
\lstinputlisting[#1]{#2}}}

% Python for inline
\newcommand\pythoninline[1]{{\pythonstyle\lstinline!#1!}}


\makeatletter
\renewcommand{\@chapapp}{Section}
\makeatother
\renewcommand{\thechapter}{\Roman{chapter}}

\pagestyle{fancy}
% \fancyhead[l]{Final Project Report \\}
\fancyhead[l]{ }
% \fancyhead[r]{Sayan Das: \texttt{B2430035} \\ Raihan Uddin: \texttt{B2430070}}
\fancyfoot[c]{\thepage}
\renewcommand{\headrulewidth}{0.2pt}
\setlength{\headheight}{28pt}

\title{Project 1 Report}
\author{
  Sayan Das
  \and
  Raihan Uddin
}

\begin{document}

\begin{titlepage}
	\newcommand{\HRule}{\rule{\linewidth}{0.5mm}}
	\center


	\textsc{\LARGE \textbf{Ramakrishna Mission Vivekananda Educational and Research Institute}}\\[1.5cm]

	\textsc{\LARGE Computer Vision}\\[0.5cm]

	\textsc{\large Project Report}\\[0.5cm]

	\HRule\\[0.4cm]

	{\huge\bfseries Image Filtering and Hybrid Images}\\[0.4cm]

	\HRule\\[1.5cm]



\large
\textit{Submitted By}\\
\textsc{\textbf{Sayan Das }}\\
\vspace{-0.5em}
\textsc{\texttt{B2430035 }}\\
\textsc{\textbf{Raihan Uddin }}\\
\vspace{-0.5em}
\textsc{\texttt{B2430070} }\\

	~
	\begin{minipage}{0.4\textwidth}
		% \begin{flushright}
		% 	\large
		% 	\textit{Submitted To}\\
		% 	\textbf{\textsc{Br. Bhaswarachaitanya (Tamal Maharaj)}}
		% \end{flushright}
	\end{minipage}


	\vfill\vfill\vfill

	{\large February 10, 2025}

	\vfill

\end{titlepage}
\restoregeometry

\tableofcontents

\chapter{Introduction}
Hybrid images, as introduced by Oliva, Torralba, and Schyns in their SIGGRAPH 2006 paper \cite{oliva2006}, are static images that change interpretation based on viewing distance. This phenomenon leverages the human visual system's multi-scale processing, where high-frequency details dominate perception at close range, while low-frequency components become prominent from afar. The goal of this project is to implement image filtering and hybrid image generation, aligning with the methodology described in the paper.
\section{Motivation}
The primary challenge in creating compelling hybrid images lies in the precise separation and recombination of these frequency components. This requires:
\vspace{-1.25em}
\begin{enumerate}
	\setlength\itemsep{-1.05em}

	\item{\textbf{Accurate Filtering: }}Implementing an image filtering algorithm that can effectively extract low and high-frequency components without introducing artifacts.
	\item{\textbf{Perceptual Alignment: }}Ensuring that the two images are aligned in a way that their frequency components blend seamlessly, avoiding perceptual conflicts.
	\item{\textbf{Parameter Tuning: }}Selecting appropriate cutoff frequencies for the filters to achieve the desired perceptual effect.

\end{enumerate}

\noindent This project aims to address these challenges by implementing a custom image filtering function and using it to generate hybrid images. The goal is to replicate the results described in the paper \cite{oliva2006} and explore the perceptual effects of hybrid images. By doing so, we aim to gain a deeper understanding of multi-scale image processing and its applications in computer vision.

\noindent The problem can be summarized as follows:

\vspace{-1.25em}
\begin{enumerate}
	\setlength\itemsep{-1.05em}

	\item{\textbf{Input: }}Two aligned images - \texttt{Image 1} and \texttt{Image 2} (e.g., \texttt{dog.bmp} and \texttt{cat.bmp}).
	\item{\textbf{Output: }}A hybrid image that changes interpretation based on viewing distance, along with intermediate results (low-frequency and high-frequency components).

\end{enumerate}


\chapter{Methodology}
\section{Image Filtering}
The core of hybrid image creation lies in filtering operations. The custom function \texttt{my\_imfilter} implements 2D convolution with the following steps:

\begin{enumerate}
	\setlength\itemsep{-1.05em}

	\item{\textbf{Padding: }}he input image is padded using zero-padding to preserve spatial dimensions post-convolution. The padding size is determined by the filter dimensions $ (k, l) $ with \label{sec:impadding}
	\begin{equation}
			\begin{aligned}
					pad_k &= \left\lfloor \frac{k}{2} \right\rfloor \\
					pad_l &= \left\lfloor \frac{l}{2} \right\rfloor
			\end{aligned}
	\end{equation}
	\item{\textbf{Convolution: }}For each color channel (RGB), a sliding window extracts image patches, which are element-wise multiplied with the filter and summed to compute the output pixel. Mathematically, \label{sec:imconvolution}
	\[
	O(x, y, c) = \sum_{i=0}^{m-1} \sum_{j=0}^{n-1} \sum_{d=0}^{D-1} I(x + i, y + j, d) \cdot K(i, j, d, c)
	\]

	where:
	\begin{itemize}
		\setlength\itemsep{-1.05em}
			\item \( O(x, y, c) \) is the output pixel value at position \( (x, y) \) for channel \( c \).
			\item \( I(x + i, y + j, d) \) is the input image pixel value at position \( (x + i, y + j) \) for channel \( d \).
			\item \( K(i, j, d, c) \) is the convolution kernel (filter) value at position \( (i, j) \), mapping from input channel \( d \) to output channel \( c \).
			\item \( m \times n \) is the size of the filter (height and width).
			\item \( D \) is the number of input channels (for RGB, \( D = 3 \)).
	\end{itemize}

\end{enumerate}


\section{Hybrid Image Generation}
The function \texttt{create\_hybrid\_image} combines two images:

\vspace{-1.25em}
\begin{enumerate}
	\setlength\itemsep{-1.05em}

	\item{\textbf{Low-Frequency Component: }}Obtained by applying a Gaussian low-pass filter to \texttt{Image 1}.
	$$ I_1 \cdot G_1 $$
	\item{\textbf{High-Frequency Component: }}Derived by subtracting the low-pass filtered version of \texttt{Image 2} from itself.
	$$ I_2 \cdot (1 - G_2) $$
	\item{\textbf{Hybrid Image: }}The sum of low and high frequencies, clipped to [0, 1] to maintain valid pixel intensities.
	$$ H = I_1 \cdot G_1 + I_2 \cdot (1 - G_2) $$

\end{enumerate}

\chapter{Implementation}
\section{Tools and Libraries}
The tools and libraries used for this project are:
\vspace{-1.25em}
\begin{enumerate}
    \setlength\itemsep{-1.05em}
    \item{\textbf{Python:}} The primary programming language for this project.
    \item{\textbf{Jupyter Notebook:}} For interactive code development.
    \item{\textbf{Libraries:}}
        \vspace{-1.5em}
        \begin{enumerate}
            \setlength\itemsep{-1.5em}
            \item{\textbf{Numpy:}} For numerical operations like array sums, and for faster mathematical computations.
						\item{\textbf{Matplotlib:}} For displaying images and visualizing filters.
						\item{\textbf{CV2:}} For image processing tasks such as reading, writing, and manipulating images.
				\end{enumerate}
\end{enumerate}

\section{Implementation Details}
\subsection{\texttt{ my\_imfilter } function}
Before constructing a hybrid image, we need a function to apply a filter to an image. The function \texttt{my\_imfilter(image, filter)} does this by:
\begin{enumerate}
	\setlength\itemsep{-1.05em}

	\item{\textbf{Padding the Image: }} Since filtering requires accessing pixels around each center pixel, we pad the input image using \pythoninline{np.pad}. The padding size is determined following the process mentioned in methodology \ref{sec:impadding}. \\ \textit{Unlike how it was mentioned in the documentation, we used k × l filter, to support rectangular kernels.}
	\item{\textbf{Applying the Filter: }} For each pixel in the image, a window of size $ (k \times l) $ is extracted.This window is element-wise multiplied with the filter and summed to get the new pixel value. This operation is repeated across all three color channels (R, G, B). \ref{sec:imconvolution}. We used \pythoninline{np.sum} for faster computation.

\end{enumerate}
The function returns the filtered image, where each pixel is computed using the convolution operation.

\subsection{\texttt{ create\_hybrid\_image } function}
The \textbf{hybrid image} is constructed by combining the low-frequency content of one image and the high-frequency content of another.

\begin{enumerate}
	\setlength\itemsep{-1.05em}
\item{Step 1: Extract Low-Frequency Content}
We obtain the \textbf{low-frequency} content of \texttt{image1} by filtering it with a low-pass filter:
\[
\text{low\_frequencies} = \text{my\_imfilter}(\text{image1}, \text{filter})
\]
A low-pass filter (e.g., a Gaussian blur) removes high-frequency details, leaving only smooth variations.
\item{Step 2: Extract High-Frequency Content}
The high-frequency content of \texttt{image2} is obtained by subtracting its low-frequency component:
\[
\text{high\_frequencies} = \text{image2} - \text{my\_imfilter}(\text{image2}, \text{filter})
\]
This step removes smooth regions and enhances sharp edges and fine details.
\item{Step 3: Combine Both Components}
Finally, the \textbf{hybrid image} is formed by adding the low-frequency and high-frequency components:
\[
\text{hybrid\_image} = \text{low\_frequencies} + \text{high\_frequencies}
\]
Since pixel values must be between \( 0 \) and \( 1 \) (for correct image representation), we apply \textbf{clipping}:
\[
\text{hybrid\_image} = \text{np.clip}(\text{hybrid\_image}, 0, 1)
\]
\end{enumerate}

The hybrid image contains \textbf{smooth variations} from \texttt{image1} and \textbf{sharp details} from \texttt{image2}. When viewed \textbf{up close}, the high-frequency details dominate (showing \texttt{image2}). When viewed \textbf{from a distance}, the low-frequency content dominates (showing \texttt{image1}).

\subsection{Creating the filter}
We used \pythoninline{cv2.getGaussianKernel(ksize, sigma)} to generate a 1D Gaussian kernel (column vector). We muliplied it with its transpose (filter @ filter.T) to convert it into a 2D Gaussian kernel. This method leverages separability, making Gaussian blurring more efficient than computing a full 2D kernel directly.

\chapter{Results}
\section{Cat-Dog Hybrid Image}
\vspace{1.25em}
\begin{figure}[H]
	\centering
	\begin{minipage}{0.45\textwidth}
			\centering
			\includegraphics[height=12em]{./images/dog.jpg}
		\end{minipage}
		\begin{minipage}{0.45\textwidth}
			\centering
			\includegraphics[height=12em]{./images/cat.jpg}
		\end{minipage}
		\caption{Original Dog and Cat Image}
		\label{dog_cat}
\end{figure}
\begin{figure}[H]
	\centering
	\begin{minipage}{0.45\textwidth}
			\centering
			\includegraphics[height=12em]{./images/dog_low.png}
		\end{minipage}
		\begin{minipage}{0.45\textwidth}
			\centering
			\includegraphics[height=12em]{./images/cat_high.png}
		\end{minipage}
		\caption{Low Frequency Dog and High Frequency Cat Image generated with cutoff-frequency=7}
		\label{dog_cat_low_high}
\end{figure}
\begin{figure}[H]
	\centering
		\includegraphics[height=15em]{./images/dog_cat_hybrid.png}
		\caption{Dog cat Hybrid Image}
		\label{dog_cat_low_hybrid}
\end{figure}
\section{Bicycle-Motorcycle Hybrid Image}
\vspace{1.25em}
\begin{figure}[H]
    \centering
    \begin{minipage}{0.45\textwidth}
            \centering
            \includegraphics[height=12em]{./images/bicycle.jpg}
        \end{minipage}
        \begin{minipage}{0.45\textwidth}
            \centering
            \includegraphics[height=12em]{./images/motorcycle.jpg}
        \end{minipage}
        \caption{Original Bicycle and Motorcycle Image}
        \label{bicycle_motorcycle}
\end{figure}
\begin{figure}[H]
    \centering
    \begin{minipage}{0.45\textwidth}
            \centering
            \includegraphics[height=12em]{./images/motorcycle_low.png}
        \end{minipage}
        \begin{minipage}{0.45\textwidth}
            \centering
            \includegraphics[height=12em]{./images/bicycle_high.png}
        \end{minipage}
        \caption{Low Frequency Bicycle and High Frequency Motorcycle Image generated with cutoff-frequency=5}
        \label{bicycle_motorcycle_low_high}
\end{figure}
\begin{figure}[H]
    \centering
        \includegraphics[height=15em]{./images/motorcycle_bicycle_hybrid.png}
        \caption{Bicycle Motorcycle Hybrid Image}
        \label{bicycle_motorcycle_hybrid}
\end{figure}

\section{Plane-Bird Hybrid Image}
\vspace{1.25em}
\begin{figure}[H]
    \centering
    \begin{minipage}{0.45\textwidth}
            \centering
            \includegraphics[height=12em]{./images/plane.jpg}
        \end{minipage}
        \begin{minipage}{0.45\textwidth}
            \centering
            \includegraphics[height=12em]{./images/bird.jpg}
        \end{minipage}
        \caption{Original Plane and Bird Image}
        \label{plane_bird}
\end{figure}
\begin{figure}[H]
    \centering
    \begin{minipage}{0.45\textwidth}
            \centering
            \includegraphics[height=12em]{./images/plane_low.png}
        \end{minipage}
        \begin{minipage}{0.45\textwidth}
            \centering
            \includegraphics[height=12em]{./images/bird_high.png}
        \end{minipage}
        \caption{Low Frequency Plane and High Frequency Bird Image generated with cutoff-frequency=7}
        \label{plane_bird_low_high}
\end{figure}
\begin{figure}[H]
    \centering
        \includegraphics[height=15em]{./images/plane_bird_hybrid.png}
        \caption{Plane Bird Hybrid Image}
        \label{plane_bird_hybrid}
\end{figure}

\section{Einstein-Marilyn Hybrid Image}
\vspace{1.25em}
\begin{figure}[H]
    \centering
    \begin{minipage}{0.45\textwidth}
            \centering
            \includegraphics[height=12em]{./images/einstein.jpg}
        \end{minipage}
        \begin{minipage}{0.45\textwidth}
            \centering
            \includegraphics[height=12em]{./images/marilyn.jpg}
        \end{minipage}
        \caption{Original Einstein and Marilyn Image}
        \label{einstein_marilyn}
\end{figure}
\begin{figure}[H]
    \centering
    \begin{minipage}{0.45\textwidth}
            \centering
            \includegraphics[height=12em]{./images/einstein_low.png}
        \end{minipage}
        \begin{minipage}{0.45\textwidth}
            \centering
            \includegraphics[height=12em]{./images/marilyn_high.png}
        \end{minipage}
        \caption{Low Frequency Einstein and High Frequency Marilyn Image generated with cutoff-frequency=3}
        \label{einstein_marilyn_low_high}
\end{figure}
\begin{figure}[H]
    \centering
        \includegraphics[height=15em]{./images/einstein_marilyn_hybrid.png}
        \caption{Einstein Marilyn Hybrid Image}
        \label{einstein_marilyn_hybrid}
\end{figure}

\section{Submarine-Fish Hybrid Image}
\vspace{1.25em}
\begin{figure}[H]
    \centering
    \begin{minipage}{0.45\textwidth}
            \centering
            \includegraphics[height=12em]{./images/submarine.jpg}
        \end{minipage}
        \begin{minipage}{0.45\textwidth}
            \centering
            \includegraphics[height=12em]{./images/fish.jpg}
        \end{minipage}
        \caption{Original Submarine and Fish Image}
        \label{submarine_fish}
\end{figure}
\begin{figure}[H]
    \centering
    \begin{minipage}{0.45\textwidth}
            \centering
            \includegraphics[height=12em]{./images/submarine_low.png}
        \end{minipage}
        \begin{minipage}{0.45\textwidth}
            \centering
            \includegraphics[height=12em]{./images/fish_high.png}
        \end{minipage}
        \caption{Low Frequency Submarine and High Frequency Fish Image generated with cutoff-frequency=4}
        \label{submarine_fish_low_high}
\end{figure}
\begin{figure}[H]
    \centering
        \includegraphics[height=15em]{./images/submarine_fish_hybrid.png}
        \caption{Submarine Fish Hybrid Image}
        \label{submarine_fish_hybrid}
\end{figure}

\chapter{Task Split}
\textbf{Raihan Uddin} was in charge of most of the implementation (coding) part. \\
\textbf{Sayan Das} was in charge of making this report in \LaTeX. He also contributed in identifying and fixing a bug in \textt{my\_imfilter} function where it was not working as intended when applying a rectangular filter. \\
\textbf{We Both} worked on reading the paper \cite{oliva2006} and helped each other understand it.

\chapter{Conclusion}
This project successfully implemented hybrid images using principles from Oliva et al.’s \cite{oliva2006} work. The results demonstrate the interplay of spatial frequencies in human perception, validating the paper’s claims. Future work could explore separable filters for speed.

\renewcommand{\bibname}{References}
\begin{thebibliography}{9}
	\bibitem{oliva2006}
  Oliva, A., Torralba, A., \& Schyns, P. G. "Hybrid Images." \textit{ACM Transactions on Graphics (TOG)}, 2006.

  \bibitem{szeliski2010}
  Szeliski, R. "Computer Vision: Algorithms and Applications." \textit{Springer}, 2010.

\end{thebibliography}
\addcontentsline{toc}{chapter}{References}

\end{document}